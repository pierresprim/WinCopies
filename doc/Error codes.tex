\documentclass[12pt]{article}
%\usepackage{longtable}
%\usepackage{ltxtable}
\usepackage{ltablex}
\usepackage{color, colortbl}
\definecolor{Gray}{gray}{0.9}
\usepackage{dcolumn}
\usepackage{titlesec}
\usepackage{array,multirow,makecell}
\setcellgapes{1pt}

\usepackage[table]{xcolor}
%\newcolumntype{R}[1]{>{\raggedleft\arraybackslash }b{#1}}
%\newcolumntype{L}[1]{>{\raggedright\arraybackslash }b{#1}}
%\newcolumntype{C}[1]{>{\centering\arraybackslash }b{#1}}
\usepackage[top=4cm, bottom=4cm, left=2cm, right=2cm]{geometry}
\usepackage[latin1]{inputenc}
\usepackage[T1]{fontenc}
\usepackage[francais]{babel}
 \usepackage{enumitem}
 \usepackage{amsmath}
\usepackage{tabularx}
\usepackage{chemfig}

\usepackage{setspace}

\usepackage{soul}

\usepackage{graphicx}

%\usepackage{fixltx2e}

\usepackage{mhchem}


\titleformat{\subsubsection}
       {\normalfont\fontfamily{phv}\fontsize{12}{17}\itshape}{\normalfont\thesubsubsection}{1em}{}

%\frenchbsetup{StandardLists=true}



\newlist{exo}{itemize}{2}
\setlist[exo,1]{label=--}
\setlist[exo,2]{label=\textbullet}

\renewcommand{\theparagraph}{}
       
       \renewcommand{\thesubparagraph}{\arabic{subparagraph}}
       
       \usepackage{multicol}
       
       \usepackage{varwidth}
       
%       \usepackage{tabulary}

\usepackage{arydshln}
       
       \newcommand{\textwidthMoinsparindent}[1]{\textwidth - #1\parindent}
       
       %\newcommand{\textwidthMoinsparindentAvecCoefficient}[1]{\textwidth - \parindent / #1}
       
       %\newcolumntype{$}{>{\global\let\currentrowstyle\relax}}
%\newcolumntype{^}{>{\currentrowstyle}}
%\newcommand{\rowstyle}[1]{\gdef\currentrowstyle{#1}%
%  #1\ignorespaces
%}

%\usepackage{wrapfig}

%\usepackage{hhline}

%\usepackage{booktabs}

%\usepackage{tabulary}

\usepackage{tabu}

%\usepackage{amsmath}

\usepackage{slashbox}

\begin{document}

\begin{center}

\Huge

\textbf{Codes d'erreurs pouvant appara�tre sous Windows}

\end{center}

\renewcommand{\tabularxcolumn}[1]{>{\arraybackslash}m{#1}}

\keepXColumns 

\begin{tabularx}{\linewidth}{ | m{5cm} | m{3cm} | X | X | }

\hline

\rowcolor{Gray} \multicolumn{1}{|c|}{Nom} & Code & Description & Impl�ment� via \\

\hline

\endhead % all the lines above this will be repeated on every page

ERROR\_FILE\par\_NOT\_FOUND & 0x2 (2) & The system cannot find the file specified. & FileNotFound \\

\hline

ERROR\_PATH\par\_NOT\_FOUND & 0x3 (3) & The system cannot find the path specified. & PathNotFound \\

\hline

ERROR\par\_ACCESS\_DENIED & 0x5 (5) & Access is denied. & AccessDenied \\

\hline

ERROR\_CURRENT\par\_DIRECTORY & 0x10 (16) & The directory cannot be removed. & DirectoryCannotBeRemoved \\

\hline

ERROR\_WRITE\par\_PROTECT & 0x13 (19) & The media is write protected. & WriteProtect \\

\hline

ERROR\par\_INVALID\_DRIVE & 0xF (15) & The system cannot find the drive specified. & \multirow{15}{*}{ExceptionOnDeviceUnit} \\

ERROR\_BAD\_UNIT & 0x14 (20) & The system cannot find the device specified. \\

ERROR\_SEEK & 0x19 (25) & The drive cannot locate a specific area or track on the disk. \\

ERROR\_NOT\_DOS\_DISK & 0x1A (26) & The specified disk or diskette cannot be accessed. \\

ERROR\_SECTOR\_NOT\_FOUND & 0x1B (27) & The drive cannot find the sector requested. \\

ERROR\_WRITE\_FAULT & 0x1D (29) & The system cannot write to the specified device. \\

ERROR\_READ\_FAULT & 0x1E (30) & The system cannot read from the specified device. \\

ERROR\_WRONG\_DISK & 0x22 (34) & The wrong diskette is in the drive. Insert \%2 (Volume Serial Number: \%3) into drive \%1. \\

\hline

ERROR\_NOT\_READY & 0x15 (21) & The device is not ready. & DiskNotReady \\

\hline

ERROR\_FILE\_EXISTS & 0x50 (80) & The file exists. & FileAlreadyExists \\

\hline

ERROR\_BUFFER\par\_OVERFLOW & 0x6F (111) & The file name is too long. & \multirow{3}{*}{FileNameTooLong} \\

ERROR\_FILENAME\par\_EXCED\_RANGE & 0xCE (206) & The filename or extension is too long. & \\

\hline

ERROR\_DISK\_FULL & 0x70 (112) & There is not enough space on the disk. & \multirow{3}{*}{DiskFull} \\

ERROR\_HANDLE\par\_DISK\_FULL & 0x27 (39) & The disk is full. & \\

\hline

ERROR\_INVALID\par\_NAME & 0x7B (123) & The filename, directory name, or volume label syntax is incorrect. & \multirow{3}{*}{InvalidName} \\

ERROR\_DIRECTORY & 0x10B (267) & The directory name is invalid. & \\

\hline

ERROR\_FILE\par\_CHECKED\_OUT & 0xDC (220) & This file is checked out or locked for editing by another user. & FileCheckedOut \\

ERROR\_SHARING\_VIOLATION & 0x20 (32) & The process cannot access the file because it is being used by another process \\

ERROR\_LOCK\_VIOLATION & 0x21 (33) & The process cannot access the file because another process has locked a portion of the file. \\

\hline

ERROR\_FILE\par\_TOO\_LARGE & 0xDF (223) & The file size exceeds the limit allowed and cannot be saved. & FileTooLarge \\

\hline

ERROR\_DISK\par\_TOO\_FRAGMENTED & 0x12E (302) & The volume is too fragmented to complete this operation. & DiskTooFragmented \\

\hline

ERROR\_DELETE\par\_PENDING & 0x12F (303) & The file cannot be opened because it is in the process of being deleted. & DeletePending \\

\hline

ERROR\_NOT\par\_ALLOWED\_ON\par\_SYSTEM\_FILE & 0x139 (313) & Operation is not allowed on a file system internal file. & NotAllowedOnSystem\par{}File \\

\hline

\end{tabularx}

\end{document}