\documentclass[12pt]{article}
%\usepackage{longtable}
%\usepackage{ltxtable}
\usepackage{ltablex}
\usepackage{color, colortbl}
\definecolor{Gray}{gray}{0.9}
\usepackage{dcolumn}
\usepackage{titlesec}
\usepackage{array,multirow,makecell}
\setcellgapes{1pt}

\usepackage[table]{xcolor}
%\newcolumntype{R}[1]{>{\raggedleft\arraybackslash }b{#1}}
%\newcolumntype{L}[1]{>{\raggedright\arraybackslash }b{#1}}
%\newcolumntype{C}[1]{>{\centering\arraybackslash }b{#1}}
\usepackage[top=4cm, bottom=4cm, left=2cm, right=2cm]{geometry}
\usepackage[latin1]{inputenc}
\usepackage[T1]{fontenc}
\usepackage[francais]{babel}
 \usepackage{enumitem}
 \usepackage{amsmath}
\usepackage{tabularx}
\usepackage{chemfig}

\usepackage{setspace}

\usepackage{soul}

\usepackage{graphicx}

%\usepackage{fixltx2e}

\usepackage{mhchem}


\titleformat{\subsubsection}
       {\normalfont\fontfamily{phv}\fontsize{12}{17}\itshape}{\normalfont\thesubsubsection}{1em}{}

%\frenchbsetup{StandardLists=true}



\newlist{exo}{itemize}{2}
\setlist[exo,1]{label=--}
\setlist[exo,2]{label=\textbullet}

\renewcommand{\theparagraph}{}
       
       \renewcommand{\thesubparagraph}{\arabic{subparagraph}}
       
       \usepackage{multicol}
       
       \usepackage{varwidth}
       
%       \usepackage{tabulary}

\usepackage{arydshln}
       
       \newcommand{\textwidthMoinsparindent}[1]{\textwidth - #1\parindent}
       
       %\newcommand{\textwidthMoinsparindentAvecCoefficient}[1]{\textwidth - \parindent / #1}
       
       %\newcolumntype{$}{>{\global\let\currentrowstyle\relax}}
%\newcolumntype{^}{>{\currentrowstyle}}
%\newcommand{\rowstyle}[1]{\gdef\currentrowstyle{#1}%
%  #1\ignorespaces
%}

%\usepackage{wrapfig}

%\usepackage{hhline}

%\usepackage{booktabs}

%\usepackage{tabulary}

\usepackage{tabu}

%\usepackage{amsmath}

\usepackage{slashbox}

\begin{document}

\begin{center}

\Huge

\textbf{WinCopies doc -- Key gestures for commands}

\end{center}

\renewcommand{\tabularxcolumn}[1]{>{\arraybackslash}m{#1}}

\keepXColumns 

\begin{tabularx}{\linewidth}{ | X | X | X!{+}X | X | }

\hline

\rowcolor{Gray} \multicolumn{1}{|c|}{Command name} & Category name & Modifier key(s) & Key & Description \\

\hline

\endhead % all the lines above this will be repeated on every page

New tab & File menu & Ctrl & T & Opens a new tab. \\

\hline

New window & File menu & Alt & W & Opens a new window. \\

\hline

New window in new instance & File menu & Ctrl + Shift & W & Opens a new window in a new instance. \\

\hline

Close tab & File menu & Ctrl & W & Close the current tab. \\

\hline

Close all tabs & File menu & Ctrl + Alt & W & Close all tabs. \\

\hline

Close window & File menu & Alt & F4 & Close the window. \\

\hline

New folder & File menu & Ctrl & N & Opens a new dialog window in order to create a new folder. \\

\hline

New archive & File menu & Ctrl + Alt & N & Opens a new dialog window in order to create a new archive. \\

\hline

Quit & File menu & Ctrl & Q & Quits the application. \\

\hline

Extract archive & N/A & Ctrl + Alt & E & Opens a dialog window in order to extract the archive. \\

\hline

Size one & View menu & Ctrl + Shift & D1 & Shows list view items using the size one. \\

\hline

Size two & View menu & Ctrl + Shift & D2 & Shows list view items using the size two. \\

\hline

Size three & View menu & Ctrl + Shift & D3 & Shows list view items using the size three. \\

\hline

Size four & View menu & Ctrl + Shift & D4 & Shows list view items using the size four. \\

\hline

List view style & View menu & Ctrl + Shift & D5 & Shows list view items using the list view style. \\

\hline

Details view style & View menu & Ctrl + Shift & D6 & Shows list view items using the details view style. \\

\hline

Mozaic view style & View menu & Ctrl + Shift & D7 & Shows list view items using the mosaic view style. \\

\hline

Content view style & View menu & Ctrl + Shift & D8 & Shows list view items using the content view style. \\

\hline

Restore tab & N/A & Ctrl + Shift & T & Restores the last closed tab. \\

\hline

Edit property & N/A & N/A & N/A & Shows a dialog box to view and edit an array property. \\

\hline

Open & List view items context menu & N/A & Enter & Opens the selected item. \\

\hline

Create shortcut & List view items context menu & Ctrl & L & Creates a new shortcut for the selected item. \\

\hline

Paste shortcut & List view items context menu & Ctrl + Alt & P & Paste the shortcut of the copied item. \\

\hline

Properties & List view items context menu & Alt & Enter & Shows the Properties dialog window. \\

\hline

\end{tabularx}

\end{document}